\section{Background}
\label{sec:background}

In this work, we focus on classification of clinical pathology reports.
Clinical pathology reports contains highly valuable information.
However, many experts have manually classified the huge volume of pathology reports and extracted information from the reports.
Also, the reports are usually unstructured and have highly varied formats because they are generated from hundreds of different medical facilities and providers.

Recently, Deep Learning (DL) based approaches in Natural Language Processing (NLP) have been applied for analysis of pathology reports and health records~\cite{yoon2016multi,miotto2016deep,qiu2017deep}.
Yoon et al.~\cite{yoon2016multi} proposed a multi-task learning model to extract important keywords from cancer pathology reports.
Moitto et al.~\cite{miotto2016deep} presented clinical predictive model using a unsupervised DL to derive patient representations that improve clinical prediction and decision system.
Recurrent Neural Networks (RNNs) have achieved a great performance in NLP tasks.
Yang et al.~\cite{yang2016hierarchical} developed a Hierarchical Attention Network (HAN) which is based on RNNs.
They designed the networks to capture a hierarchical structure of documents (words form sentences, sentences form a document).
HAN contains two levels of attention mechanisms to extract important words and sentences in a document.
% HAN utilize a softmax function for document classification.
In this work we use a set of model snapshot data resulted from the hierarchical attention layers and the final softmax output layer in HAN.
